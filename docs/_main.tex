% Options for packages loaded elsewhere
\PassOptionsToPackage{unicode}{hyperref}
\PassOptionsToPackage{hyphens}{url}
%
\documentclass[
]{book}
\usepackage{amsmath,amssymb}
\usepackage{iftex}
\ifPDFTeX
  \usepackage[T1]{fontenc}
  \usepackage[utf8]{inputenc}
  \usepackage{textcomp} % provide euro and other symbols
\else % if luatex or xetex
  \usepackage{unicode-math} % this also loads fontspec
  \defaultfontfeatures{Scale=MatchLowercase}
  \defaultfontfeatures[\rmfamily]{Ligatures=TeX,Scale=1}
\fi
\usepackage{lmodern}
\ifPDFTeX\else
  % xetex/luatex font selection
\fi
% Use upquote if available, for straight quotes in verbatim environments
\IfFileExists{upquote.sty}{\usepackage{upquote}}{}
\IfFileExists{microtype.sty}{% use microtype if available
  \usepackage[]{microtype}
  \UseMicrotypeSet[protrusion]{basicmath} % disable protrusion for tt fonts
}{}
\makeatletter
\@ifundefined{KOMAClassName}{% if non-KOMA class
  \IfFileExists{parskip.sty}{%
    \usepackage{parskip}
  }{% else
    \setlength{\parindent}{0pt}
    \setlength{\parskip}{6pt plus 2pt minus 1pt}}
}{% if KOMA class
  \KOMAoptions{parskip=half}}
\makeatother
\usepackage{xcolor}
\usepackage{color}
\usepackage{fancyvrb}
\newcommand{\VerbBar}{|}
\newcommand{\VERB}{\Verb[commandchars=\\\{\}]}
\DefineVerbatimEnvironment{Highlighting}{Verbatim}{commandchars=\\\{\}}
% Add ',fontsize=\small' for more characters per line
\usepackage{framed}
\definecolor{shadecolor}{RGB}{248,248,248}
\newenvironment{Shaded}{\begin{snugshade}}{\end{snugshade}}
\newcommand{\AlertTok}[1]{\textcolor[rgb]{0.94,0.16,0.16}{#1}}
\newcommand{\AnnotationTok}[1]{\textcolor[rgb]{0.56,0.35,0.01}{\textbf{\textit{#1}}}}
\newcommand{\AttributeTok}[1]{\textcolor[rgb]{0.13,0.29,0.53}{#1}}
\newcommand{\BaseNTok}[1]{\textcolor[rgb]{0.00,0.00,0.81}{#1}}
\newcommand{\BuiltInTok}[1]{#1}
\newcommand{\CharTok}[1]{\textcolor[rgb]{0.31,0.60,0.02}{#1}}
\newcommand{\CommentTok}[1]{\textcolor[rgb]{0.56,0.35,0.01}{\textit{#1}}}
\newcommand{\CommentVarTok}[1]{\textcolor[rgb]{0.56,0.35,0.01}{\textbf{\textit{#1}}}}
\newcommand{\ConstantTok}[1]{\textcolor[rgb]{0.56,0.35,0.01}{#1}}
\newcommand{\ControlFlowTok}[1]{\textcolor[rgb]{0.13,0.29,0.53}{\textbf{#1}}}
\newcommand{\DataTypeTok}[1]{\textcolor[rgb]{0.13,0.29,0.53}{#1}}
\newcommand{\DecValTok}[1]{\textcolor[rgb]{0.00,0.00,0.81}{#1}}
\newcommand{\DocumentationTok}[1]{\textcolor[rgb]{0.56,0.35,0.01}{\textbf{\textit{#1}}}}
\newcommand{\ErrorTok}[1]{\textcolor[rgb]{0.64,0.00,0.00}{\textbf{#1}}}
\newcommand{\ExtensionTok}[1]{#1}
\newcommand{\FloatTok}[1]{\textcolor[rgb]{0.00,0.00,0.81}{#1}}
\newcommand{\FunctionTok}[1]{\textcolor[rgb]{0.13,0.29,0.53}{\textbf{#1}}}
\newcommand{\ImportTok}[1]{#1}
\newcommand{\InformationTok}[1]{\textcolor[rgb]{0.56,0.35,0.01}{\textbf{\textit{#1}}}}
\newcommand{\KeywordTok}[1]{\textcolor[rgb]{0.13,0.29,0.53}{\textbf{#1}}}
\newcommand{\NormalTok}[1]{#1}
\newcommand{\OperatorTok}[1]{\textcolor[rgb]{0.81,0.36,0.00}{\textbf{#1}}}
\newcommand{\OtherTok}[1]{\textcolor[rgb]{0.56,0.35,0.01}{#1}}
\newcommand{\PreprocessorTok}[1]{\textcolor[rgb]{0.56,0.35,0.01}{\textit{#1}}}
\newcommand{\RegionMarkerTok}[1]{#1}
\newcommand{\SpecialCharTok}[1]{\textcolor[rgb]{0.81,0.36,0.00}{\textbf{#1}}}
\newcommand{\SpecialStringTok}[1]{\textcolor[rgb]{0.31,0.60,0.02}{#1}}
\newcommand{\StringTok}[1]{\textcolor[rgb]{0.31,0.60,0.02}{#1}}
\newcommand{\VariableTok}[1]{\textcolor[rgb]{0.00,0.00,0.00}{#1}}
\newcommand{\VerbatimStringTok}[1]{\textcolor[rgb]{0.31,0.60,0.02}{#1}}
\newcommand{\WarningTok}[1]{\textcolor[rgb]{0.56,0.35,0.01}{\textbf{\textit{#1}}}}
\usepackage{longtable,booktabs,array}
\usepackage{calc} % for calculating minipage widths
% Correct order of tables after \paragraph or \subparagraph
\usepackage{etoolbox}
\makeatletter
\patchcmd\longtable{\par}{\if@noskipsec\mbox{}\fi\par}{}{}
\makeatother
% Allow footnotes in longtable head/foot
\IfFileExists{footnotehyper.sty}{\usepackage{footnotehyper}}{\usepackage{footnote}}
\makesavenoteenv{longtable}
\usepackage{graphicx}
\makeatletter
\def\maxwidth{\ifdim\Gin@nat@width>\linewidth\linewidth\else\Gin@nat@width\fi}
\def\maxheight{\ifdim\Gin@nat@height>\textheight\textheight\else\Gin@nat@height\fi}
\makeatother
% Scale images if necessary, so that they will not overflow the page
% margins by default, and it is still possible to overwrite the defaults
% using explicit options in \includegraphics[width, height, ...]{}
\setkeys{Gin}{width=\maxwidth,height=\maxheight,keepaspectratio}
% Set default figure placement to htbp
\makeatletter
\def\fps@figure{htbp}
\makeatother
\setlength{\emergencystretch}{3em} % prevent overfull lines
\providecommand{\tightlist}{%
  \setlength{\itemsep}{0pt}\setlength{\parskip}{0pt}}
\setcounter{secnumdepth}{5}
\usepackage{booktabs}
\ifLuaTeX
  \usepackage{selnolig}  % disable illegal ligatures
\fi
\usepackage[]{natbib}
\bibliographystyle{plainnat}
\IfFileExists{bookmark.sty}{\usepackage{bookmark}}{\usepackage{hyperref}}
\IfFileExists{xurl.sty}{\usepackage{xurl}}{} % add URL line breaks if available
\urlstyle{same}
\hypersetup{
  pdftitle={一份卫生统计学笔记},
  pdfauthor={Zhoulvbang},
  hidelinks,
  pdfcreator={LaTeX via pandoc}}

\title{一份卫生统计学笔记}
\author{Zhoulvbang}
\date{2024-03-18}

\usepackage{amsthm}
\newtheorem{theorem}{Theorem}[chapter]
\newtheorem{lemma}{Lemma}[chapter]
\newtheorem{corollary}{Corollary}[chapter]
\newtheorem{proposition}{Proposition}[chapter]
\newtheorem{conjecture}{Conjecture}[chapter]
\theoremstyle{definition}
\newtheorem{definition}{Definition}[chapter]
\theoremstyle{definition}
\newtheorem{example}{Example}[chapter]
\theoremstyle{definition}
\newtheorem{exercise}{Exercise}[chapter]
\theoremstyle{definition}
\newtheorem{hypothesis}{Hypothesis}[chapter]
\theoremstyle{remark}
\newtheorem*{remark}{Remark}
\newtheorem*{solution}{Solution}
\begin{document}
\maketitle

{
\setcounter{tocdepth}{1}
\tableofcontents
}
\hypertarget{aboutux5173ux4e8e}{%
\chapter{About(关于)}\label{aboutux5173ux4e8e}}

这是一个用\textbf{Markdown}编写的 \emph{小周的学习笔记} 书籍。您可以使用Pandoc的Markdown支持的任何内容;例如,数学方程式 \(a^2 + b^2 = c^2\)。

``All knowledge is, in final analysis, history.

All sciences are, in the abstact, mathematics.

All judgements are, in their rationale, statistics.''

``在终极的分析中,一切知识都是历史

在抽象的意义下,一切科学都是数学

在理性的基础上,所有的判断都是统计学''

------C.R. Rao \emph{Statistics And Truth}

\hypertarget{usageux7528ux6cd5}{%
\section{Usage(用法)}\label{usageux7528ux6cd5}}

每个 \textbf{bookdown} 章节都是一个 \texttt{.Rmd} 文件,每个 \texttt{.Rmd} 文件只能包含一个章节。一个章节 \emph{必须} 以一级标题开始:\texttt{\#\ 一个好的章节},并且可以包含一个(且只能有一个)一级标题。

在章节内部使用二级标题及更高级别的标题,例如:\texttt{\#\#\ 一个简短的小节} 或 \texttt{\#\#\#\ 一个更短的小节}。

\texttt{index.Rmd} 文件是必需的,并且也是您的第一章。当您渲染这本书时,它将成为主页。

\hypertarget{render-bookux6e32ux67d3ux8fd9ux672cux4e66}{%
\section{Render book(渲染这本书)}\label{render-bookux6e32ux67d3ux8fd9ux672cux4e66}}

您可以在不更改任何内容的情况下,渲染此示例书籍的 HTML 版本:

\begin{enumerate}
\def\labelenumi{\arabic{enumi}.}
\item
  在 RStudio IDE 中找到 \textbf{Build} 面板,然后
\item
  单击 \textbf{Build Book},然后选择您的输出格式,或者如果您想要从相同的书籍源文件中使用多个格式,则选择 ``All formats''。
\end{enumerate}

或者从 R 控制台构建书籍:

\begin{Shaded}
\begin{Highlighting}[]
\NormalTok{bookdown}\SpecialCharTok{::}\FunctionTok{render\_book}\NormalTok{()}
\end{Highlighting}
\end{Shaded}

要将此示例渲染为 \texttt{bookdown::pdf\_book},您需要安装 \(XeLaTeX\)。建议您安装 TinyTeX(其中包含 \(XeLaTeX\)):\url{https://yihui.org/tinytex/}。

当然你也可以下载别的\(TeX\)发行版,比如\href{https://tug.org/texlive/}{TeXLive}和\href{https://tug.org/mactex/}{MaCTeX}。对于\texttt{TeXLive}的下载可以参照\href{https://zhuanlan.zhihu.com/p/362201376}{LaTeX:TeXLive2021安装}来进行;\texttt{MaCTeX}则可以参照\href{}{}

\hypertarget{preview-bookux9884ux89c8ux8fd9ux672cux4e66}{%
\section{Preview book(预览这本书)}\label{preview-bookux9884ux89c8ux8fd9ux672cux4e66}}

在您工作时,您可以启动本地服务器以实时预览这本 HTML 书籍。当您保存单独的 \texttt{.Rmd} 文件时,此预览将会更新。您可以通过使用 RStudio 的插件 ``Preview book'' 或从 R 控制台中启动服务器。

\begin{Shaded}
\begin{Highlighting}[]
\NormalTok{bookdown}\SpecialCharTok{::}\FunctionTok{serve\_book}\NormalTok{()}
\end{Highlighting}
\end{Shaded}

\hypertarget{ux4e0ermarkdownux7684ux7f18ux5206}{%
\section{与Rmarkdown的缘分}\label{ux4e0ermarkdownux7684ux7f18ux5206}}

有很多方法制作一个学术主页/网站,我是过很多办法,最开始是使用\texttt{gridea},那还是大二的故事,看见一个朋友搭建了一夜主页,我很好奇,也觉得很好玩(其实是逼格很高),我就觉得我也得来一个,奈何不是学计算机的,很多原理都没搞明白,网络问题也是很棘手,弄着弄着就没兴趣了,然后荒废了,当时也确实没什么东西可以写,纯粹就是为了装逼而装逼。

后来到了大四的时候,我看到\href{https://academicpages.github.io/}{Academic page},我就又觉得我行了,结果还是史无前例的大坑啊!因为这个网页的搭建是使用\texttt{jerkll}来生成静态网页,需要本地对其进行渲染然后在Internet中显示,我前期只是修改了一些信息然后就将其放置在GitHub上,然后在腾讯云租用了一个域名,将其关联(其实22年租用了域名以后,一直没搞好,专业知识太匮乏,没得办法\textasciitilde)。

关于这个坑啊,有时间再专门写一个post来讲\textasciitilde{}

后来学习准备二战,学着学着其实有点无聊,你懂吧\textasciitilde{}

就是啥都可以弄一弄,但是就是对专业课没有心思了,所以吧,我捉摸着学一学\texttt{R},然后就一发不可收拾啊,又让我发现了\texttt{Rmarkdown}这个好东西,嘿嘿,用来做笔记其实挺爽的,刚好卫生统计学有不少公式和图标,都可以不错的解决\textasciitilde{}

这里就不是不说一下另外几个硬货的优缺点了,敲黑板···

\textbf{MSOFFICE}:微软的办公套装,Word没啥不好,就是公式难得处理,还是文本格式比较难以管控,一不小心就是划勒巴子散架了,成也灵活,败也灵活。

\textbf{LaTeX}:这个东西也是花了我很多时间去学的,特别是大一大二的时候,被高数老师说的数学建模比赛迷得七荤八素,然后高数老师在大一的寒假带着我们学习\(LaTeX\)这个啊,说我们先学会这个,其他的等开学了我再教,结果啊,碰上疫情,吹了,数学课题组没人管这个了,那好吧,我就自己玩。\(LaTeX\)就是很规矩,很硬,一板一眼,有自己独特的一套逻辑与美感,成也规矩,败也规矩。

\textbf{Ramrkdown}:受益于xieyihui的天才般的构想,将\texttt{markdown}、\texttt{LaTeX}与\texttt{html}结合起来,将\texttt{markdown}的轻量化与简洁同\(LaTeX\)的规矩与公式显示结合起来,再结合R对数据数据与分析的强大能力,形成了现在的\texttt{Rmarkdown}。

\hypertarget{ux5173ux4e8er}{%
\section{关于R}\label{ux5173ux4e8er}}

R语言是一种自由软件编程语言与操作环境,主要用于统计分析、绘图以及数据挖掘。R由新西兰奥克兰大学的统计学家罗斯·伊哈卡和罗伯特·杰特曼开发,现在由R核心小组负责开发,同时也有其他用户编写了诸多外挂的软件包。R以S语言为基础,其词法作用域语义来自Scheme。R的后台程序大多由C语言、FORTRAN语言和R语言自己写成。

\hypertarget{ux5b89ux88c5rrstudio}{%
\section{安装R+Rstudio}\label{ux5b89ux88c5rrstudio}}

回归正题,要使用\texttt{rmarkdown}需要我们使用\texttt{R}与\texttt{Rstudio},我们可以从\href{https://www.r-project.org/}{R}来下载\texttt{R}这个程序,
\texttt{R} 是一个用于统计计算和图形的免费软件环境;再从\href{https://www.rstudio.com/categories/rstudio-ide/}{Rstudio}来下载适配R的集成开发编辑器,普通的\texttt{R\ script}可以在\texttt{vscode}中进行编辑和运行,但是\texttt{rmarkdown}文件则需要再\texttt{Rstudio}中才能较好的编辑与运行。

安装的步骤可以参考\href{https://zhuanlan.zhihu.com/p/687529684}{在macOS中安装R与Rstudio}

\textbf{Notice}:macOS需要看一下是Intel还是arm的CPU,arm目前在M1/M2上搭载

\hypertarget{ux521bux5efaux4e00ux4e2armarkdownux6587ux4ef6}{%
\section{创建一个Rmarkdown文件}\label{ux521bux5efaux4e00ux4e2armarkdownux6587ux4ef6}}

因为我们要将这个文件配置到GitHub上作为网页,所以我们要创建一个特别一点的rmarkdown。

\hypertarget{ux9996ux5148ux6211ux4eecux8981ux5b89ux88c5ux4e00ux4e2aux4e13ux95e8ux7684ux5305bookdown}{%
\subsection{\texorpdfstring{首先我们要安装一个专门的包\texttt{bookdown}}{首先我们要安装一个专门的包bookdown}}\label{ux9996ux5148ux6211ux4eecux8981ux5b89ux88c5ux4e00ux4e2aux4e13ux95e8ux7684ux5305bookdown}}

\begin{Shaded}
\begin{Highlighting}[]
\FunctionTok{install.packages}\NormalTok{(}\StringTok{\textquotesingle{}bookdown\textquotesingle{}}\NormalTok{)}
\end{Highlighting}
\end{Shaded}

\hypertarget{ux521bux5efaux4e00ux4e2abookdownux6587ux4ef6}{%
\subsection{\texorpdfstring{创建一个\texttt{bookdown}文件}{创建一个bookdown文件}}\label{ux521bux5efaux4e00ux4e2abookdownux6587ux4ef6}}

\begin{enumerate}
\def\labelenumi{\arabic{enumi}.}
\tightlist
\item
  create project
\end{enumerate}

在Rstudio中点击File-\textgreater New Project-\textgreater New Directory-\textgreater Bookproject using bookdown-\textgreater Create Project

\hypertarget{ux8be6ux89e3ux6613ux6df7ux6dc6ux7684ux75beux75c5ux6d4bux91cfux6307ux6807}{%
\chapter{详解易混淆的疾病测量指标}\label{ux8be6ux89e3ux6613ux6df7ux6dc6ux7684ux75beux75c5ux6d4bux91cfux6307ux6807}}

\hypertarget{ux77e5ux8bc6ux70b9ux5b9aux4f4dux75beux75c5ux9891ux7387ux6d4bux91cf}{%
\section{知识点定位------疾病频率测量}\label{ux77e5ux8bc6ux70b9ux5b9aux4f4dux75beux75c5ux9891ux7387ux6d4bux91cf}}

\hypertarget{ux53d1ux75c5}{%
\subsection{发病}\label{ux53d1ux75c5}}

\begin{itemize}
\tightlist
\item
  发病率------流行强度
\item
  罹患率------暴露程度
\item
  续发率------传染病
\end{itemize}

\hypertarget{ux60a3ux75c5}{%
\subsection{患病}\label{ux60a3ux75c5}}

\begin{itemize}
\tightlist
\item
  患病率------慢性病
\item
  感染率------传染病
\end{itemize}

\hypertarget{ux6b7bux4ea1}{%
\subsection{死亡}\label{ux6b7bux4ea1}}

\begin{itemize}
\tightlist
\item
  死亡率------死亡风险
\item
  病死率------急性传染病/慢性病
\item
  婴儿死亡率------发展水平
\item
  生存率------慢性病
\end{itemize}

\hypertarget{ux75beux75c5ux8d1fux62c5}{%
\subsection{疾病负担}\label{ux75beux75c5ux8d1fux62c5}}

\begin{itemize}
\tightlist
\item
  PYLL------死亡造成的寿命损失
\item
  DALY------发病到死亡损失的全部健康寿命
\end{itemize}

\hypertarget{ux5b9aux4e49}{%
\section{定义}\label{ux5b9aux4e49}}

\begin{itemize}
\tightlist
\item
  发病率(Incidence Rate):单位时间内,,通常以每1000或每10万人计算。
\item
  患病率(Prevalence Rate):也称现患率,特定时间点或一段时间内某病新旧病例占总人口的比例。

  \begin{itemize}
  \tightlist
  \item
    分为时点患病率(\textless1个月)和期间患病率(几个月)
  \end{itemize}
\item
  新发病率(Cumulative Incidence): 特定时间段内发生新病例的比例,通常以百分比表示。
\item
  死亡率(Death Rate or Mortality Rate):在一定时期内,某人群中总死亡人数在该人群中所占的比例,测量人群死亡危险最常用的指标
\item
  病死率(Case Fatality Rate):一定时期内,因某病死亡者占该病病人的比例,表示某病病人因该病死亡的危险性
\end{itemize}

\hypertarget{ux53d1ux75c5ux7387vsux60a3ux75c5ux7387}{%
\section{发病率VS患病率}\label{ux53d1ux75c5ux7387vsux60a3ux75c5ux7387}}

\begin{longtable}[]{@{}
  >{\raggedright\arraybackslash}p{(\columnwidth - 4\tabcolsep) * \real{0.2000}}
  >{\raggedright\arraybackslash}p{(\columnwidth - 4\tabcolsep) * \real{0.4000}}
  >{\raggedright\arraybackslash}p{(\columnwidth - 4\tabcolsep) * \real{0.4000}}@{}}
\toprule\noalign{}
\begin{minipage}[b]{\linewidth}\raggedright
分类
\end{minipage} & \begin{minipage}[b]{\linewidth}\raggedright
发病率
\end{minipage} & \begin{minipage}[b]{\linewidth}\raggedright
患病率
\end{minipage} \\
\midrule\noalign{}
\endhead
\bottomrule\noalign{}
\endlastfoot
\textbf{资料来源} & 疾病报告/监测/队列研究 & 现况调查 \\
\textbf{计算分子} & 观察期新发病例数 & 观察期新旧病例数 \\
\textbf{计算分母} & 平均人口数/暴露人口数 & 调查人数/平均人口数 \\
\textbf{观察时间} & 一般1年/更长 & 较短,1/几个月 \\
\textbf{适用疾病} & 各种疾病 & 慢性病/病程较长疾病 \\
\textbf{特点} & 动态描述 & 静态描述 \\
\textbf{用途} & 疾病流行强度 & 疾病现患状况/慢性病流行情况 \\
\textbf{影响因素} & 较少:疾病流行情况/诊断水平/疾病报告质量 & 较多:影响发病率的因素/病后死亡或痊愈及康复状况病程等 \\
\end{longtable}

\hypertarget{ux62bdux6837}{%
\chapter{抽样}\label{ux62bdux6837}}

\hypertarget{ux5e38ux89c1ux7684ux6982ux7387ux62bdux6837}{%
\section{常见的概率抽样}\label{ux5e38ux89c1ux7684ux6982ux7387ux62bdux6837}}

\begin{longtable}[]{@{}
  >{\raggedright\arraybackslash}p{(\columnwidth - 10\tabcolsep) * \real{0.0909}}
  >{\raggedright\arraybackslash}p{(\columnwidth - 10\tabcolsep) * \real{0.1818}}
  >{\raggedright\arraybackslash}p{(\columnwidth - 10\tabcolsep) * \real{0.1818}}
  >{\raggedright\arraybackslash}p{(\columnwidth - 10\tabcolsep) * \real{0.1818}}
  >{\raggedright\arraybackslash}p{(\columnwidth - 10\tabcolsep) * \real{0.1818}}
  >{\raggedright\arraybackslash}p{(\columnwidth - 10\tabcolsep) * \real{0.1818}}@{}}
\toprule\noalign{}
\begin{minipage}[b]{\linewidth}\raggedright
类别
\end{minipage} & \begin{minipage}[b]{\linewidth}\raggedright
简单随机抽样
\end{minipage} & \begin{minipage}[b]{\linewidth}\raggedright
系统抽样
\end{minipage} & \begin{minipage}[b]{\linewidth}\raggedright
整群抽样
\end{minipage} & \begin{minipage}[b]{\linewidth}\raggedright
分层抽样
\end{minipage} & \begin{minipage}[b]{\linewidth}\raggedright
多阶段抽样
\end{minipage} \\
\midrule\noalign{}
\endhead
\bottomrule\noalign{}
\endlastfoot
概念 & 将全部的观察单位编号,形成抽样框,在抽样框中随机抽取部分观察单位组成样本 & 先将总体的观察单位按照某一顺序分成n个部分,再从第一部分随机抽取第k号观察单位,依次用相等间隔,从每一部分各抽取一个观察单位组成样本 & 是以``群''为基本单位的抽样方法,先将总体分成若干群,从中随机抽取一些群,被抽中群内的全部个体组成调查的样本 & 先将总体中全部个体按某种特征分成若干``层'',再从每一层内随机抽取一定数量的个体组成样本 & 将整个抽样过程分成若干阶段进行,在初级抽样单位中抽取二级抽样单位,又在二级抽样单位中抽取三级抽样单位 \\
优点 & 简单直观;均数(率)及其标准误计算简便 & 易于理解、简便易行;可得到按比例分配的样本;样本在总体中的分布均匀 & 便于组织调查;节约成本;容易控制调查质量 & 抽样误差相对较小;可对不同层采用不同的抽样方法;可对不同层进行独立分析 & 充分利用各种抽样方法的优势,克服各自的不足,并能节省人力、物力 \\
缺点 & 观察单位较多,编号在实际工作中难以实现;当总体变异大时,抽样误差较分层抽样误差大 & 观察单位按顺序有周期趋势或递增(减)时易产生偏差 & 样本例数一定时,抽样误差大于简单随机抽样(因样本为广泛散布于总体中 & 若分层变量选择不当,层内变异较大,层间变异较小,则分层抽样失去意义 & 在抽样之前要掌握各级调查单位的人口资料及特点 \\
适用范围 & 是其他抽样方法的基础,主要用于总体不太大的情形 & 主要用于按抽样顺序个体随机分布的情形 & 主要用于群间差异较小的情形 & 主要用于层间差异较大的情形 & 大型流行病学调查 \\
\end{longtable}

\textbf{误差大小:} 整群抽样\textgreater 简单随机抽样\textgreater 系统抽样\textgreater 分层抽样

\textbf{样本量大小:}整群抽样\textgreater 简单随机抽样\textgreater 系统抽样\textgreater 分层抽样

\textbf{概率抽样:}是指每个个体被抽样抽中的概率是非零的、已知的或可计算的。

\hypertarget{ux5e38ux89c1ux7684ux975eux6982ux7387ux62bdux6837}{%
\section{常见的非概率抽样}\label{ux5e38ux89c1ux7684ux975eux6982ux7387ux62bdux6837}}

\begin{itemize}
\tightlist
\item
  特点

  \begin{itemize}
  \tightlist
  \item
    不需要考虑等概率原则
  \item
    依赖研究人员的经验和专业知识
  \item
    简便易行、节约资源
  \item
    结果的稳定性容易受主观影响
  \end{itemize}
\end{itemize}

\begin{longtable}[]{@{}
  >{\centering\arraybackslash}p{(\columnwidth - 2\tabcolsep) * \real{0.3333}}
  >{\centering\arraybackslash}p{(\columnwidth - 2\tabcolsep) * \real{0.6667}}@{}}
\toprule\noalign{}
\begin{minipage}[b]{\linewidth}\centering
类别
\end{minipage} & \begin{minipage}[b]{\linewidth}\centering
概念
\end{minipage} \\
\midrule\noalign{}
\endhead
\bottomrule\noalign{}
\endlastfoot
偶遇抽样 & 又称便利抽样,指研究者根据实际情况而采用最便利的方法来选取样本,可以抽取偶然遇到的人,或选择那些距离最近的、最容易找到的人作为调查对象 \\
目的抽样 & 又称判断抽样,指研究者根据研究目标和对情况的主观判断来选择和确定调查对象的方法,是``有目的''地去选择对总体具有代表性的样本 \\
滚雪球抽样 & 又称链式抽样或网络抽样,指当无法了解总体情形时,可以从能找到的少数个体入手,对他们进行调查,并请他们介绍其他符合条件的人,扩大调查面,如此重复下去直到达到所需的样本量 \\
定额抽样 & 又称配额抽样,是按照总体的某种特征(年龄、性别、社会阶层等)进行分层(组),然后在每一层(组)中按照事先规定的比例或数量(即定额)用便利抽样或目的抽样的方法选取样本 \\
空间抽样 & 指对具有空间关联性的各种调查对象及资源进行抽样的一种方法 \\
\end{longtable}

\hypertarget{ux5173ux4e8eux536bux751fux7edfux8ba1ux5b66}{%
\chapter{关于卫生统计学}\label{ux5173ux4e8eux536bux751fux7edfux8ba1ux5b66}}

\hypertarget{ux536bux751fux7edfux8ba1ux5b66ux4ecbux7ecdux4e0eux5c55ux671b}{%
\section{卫生统计学:介绍与展望}\label{ux536bux751fux7edfux8ba1ux5b66ux4ecbux7ecdux4e0eux5c55ux671b}}

\hypertarget{ux4ec0ux4e48ux662fux536bux751fux7edfux8ba1ux5b66}{%
\subsection{什么是卫生统计学?}\label{ux4ec0ux4e48ux662fux536bux751fux7edfux8ba1ux5b66}}

卫生统计学是一门致力于收集、分析和解释与健康相关的数据的学科。它的目标是通过统计方法来评估和改善人群的健康状况,从而提高公共卫生水平。卫生统计学将统计学原理应用于医学和公共卫生领域,以支持健康决策的制定和实施。

\hypertarget{ux536bux751fux7edfux8ba1ux5b66ux7684ux91cdux8981ux6027}{%
\subsection{卫生统计学的重要性}\label{ux536bux751fux7edfux8ba1ux5b66ux7684ux91cdux8981ux6027}}

卫生统计学在以下几个方面发挥着重要作用:

\begin{enumerate}
\def\labelenumi{\arabic{enumi}.}
\item
  \textbf{疾病监测与控制:} 通过收集和分析疾病发病率、死亡率和流行病学数据,卫生统计学帮助识别疾病的流行趋势,并制定相应的预防和控制策略。
\item
  \textbf{健康政策制定:} 卫生统计学提供了评估不同健康政策和干预措施效果的方法,为政策制定者提供了决策支持。
\item
  \textbf{卫生服务评估:} 通过分析卫生服务的覆盖范围、质量和效率,卫生统计学评估卫生系统的运作情况,并提供改进建议。
\item
  \textbf{流行病学研究:} 卫生统计学在研究人群健康与疾病之间的关系方面发挥着关键作用,帮助揭示疾病的发病机制和影响因素。
\end{enumerate}

\hypertarget{ux536bux751fux7edfux8ba1ux5b66ux7684ux672aux6765ux5c55ux671b}{%
\subsection{卫生统计学的未来展望}\label{ux536bux751fux7edfux8ba1ux5b66ux7684ux672aux6765ux5c55ux671b}}

随着数据科学和人工智能技术的发展,卫生统计学将迎来新的机遇和挑战:

\begin{enumerate}
\def\labelenumi{\arabic{enumi}.}
\item
  \textbf{大数据与人工智能:} 大数据技术使得收集、整合和分析海量的健康数据成为可能,而人工智能技术则提供了更高效、精确的数据处理和预测能力,为卫生统计学研究提供了新的方法和工具。
\item
  \textbf{个性化医疗:} 基于个体遗传信息和生活方式数据的个性化医疗将成为未来的发展趋势,卫生统计学将在个体化医疗决策和健康管理中发挥更加重要的作用。
\item
  \textbf{跨学科合作:} 卫生统计学将与流行病学、遗传学、生物信息学等学科交叉融合,形成跨学科合作的新模式,共同解决健康领域的复杂问题。
\item
  \textbf{公众参与与健康促进:} 卫生统计学将更加注重公众参与和社区健康促进,通过社会行为和环境因素的分析,促进健康政策的制定和实施,推动社会健康公平。
\end{enumerate}

\hypertarget{ux7ed3ux8bed}{%
\subsection{结语}\label{ux7ed3ux8bed}}

随着社会的发展和健康需求的不断变化,卫生统计学将继续发挥着重要的作用,为促进全民健康、预防疾病、改善医疗服务质量和推动卫生政策提供支持和指导。

\hypertarget{parts}{%
\chapter{Parts}\label{parts}}

You can add parts to organize one or more book chapters together. Parts can be inserted at the top of an .Rmd file, before the first-level chapter heading in that same file.

Add a numbered part: \texttt{\#\ (PART)\ Act\ one\ \{-\}} (followed by \texttt{\#\ A\ chapter})

Add an unnumbered part: \texttt{\#\ (PART\textbackslash{}*)\ Act\ one\ \{-\}} (followed by \texttt{\#\ A\ chapter})

Add an appendix as a special kind of un-numbered part: \texttt{\#\ (APPENDIX)\ Other\ stuff\ \{-\}} (followed by \texttt{\#\ A\ chapter}). Chapters in an appendix are prepended with letters instead of numbers.

\hypertarget{footnotes-and-citations}{%
\chapter{Footnotes and citations}\label{footnotes-and-citations}}

\hypertarget{footnotes}{%
\section{Footnotes}\label{footnotes}}

Footnotes are put inside the square brackets after a caret \texttt{\^{}{[}{]}}. Like this one \footnote{This is a footnote.}.

\hypertarget{citations}{%
\section{Citations}\label{citations}}

Reference items in your bibliography file(s) using \texttt{@key}.

For example, we are using the \textbf{bookdown} package \citep{R-bookdown} (check out the last code chunk in index.Rmd to see how this citation key was added) in this sample book, which was built on top of R Markdown and \textbf{knitr} \citep{xie2015} (this citation was added manually in an external file book.bib).
Note that the \texttt{.bib} files need to be listed in the index.Rmd with the YAML \texttt{bibliography} key.

The RStudio Visual Markdown Editor can also make it easier to insert citations: \url{https://rstudio.github.io/visual-markdown-editing/\#/citations}

\hypertarget{blocks}{%
\chapter{Blocks}\label{blocks}}

\hypertarget{equations}{%
\section{Equations}\label{equations}}

Here is an equation.

\begin{equation} 
  f\left(k\right) = \binom{n}{k} p^k\left(1-p\right)^{n-k}
  \label{eq:binom}
\end{equation}

You may refer to using \texttt{\textbackslash{}@ref(eq:binom)}, like see Equation \eqref{eq:binom}.

\hypertarget{theorems-and-proofs}{%
\section{Theorems and proofs}\label{theorems-and-proofs}}

Labeled theorems can be referenced in text using \texttt{\textbackslash{}@ref(thm:tri)}, for example, check out this smart theorem \ref{thm:tri}.

\begin{theorem}
\protect\hypertarget{thm:tri}{}\label{thm:tri}For a right triangle, if \(c\) denotes the \emph{length} of the hypotenuse
and \(a\) and \(b\) denote the lengths of the \textbf{other} two sides, we have
\[a^2 + b^2 = c^2\]
\end{theorem}

Read more here \url{https://bookdown.org/yihui/bookdown/markdown-extensions-by-bookdown.html}.

\hypertarget{callout-blocks}{%
\section{Callout blocks}\label{callout-blocks}}

The R Markdown Cookbook provides more help on how to use custom blocks to design your own callouts: \url{https://bookdown.org/yihui/rmarkdown-cookbook/custom-blocks.html}

\hypertarget{sharing-your-book}{%
\chapter{Sharing your book}\label{sharing-your-book}}

\hypertarget{publishing}{%
\section{Publishing}\label{publishing}}

HTML books can be published online, see: \url{https://bookdown.org/yihui/bookdown/publishing.html}

\hypertarget{pages}{%
\section{404 pages}\label{pages}}

By default, users will be directed to a 404 page if they try to access a webpage that cannot be found. If you'd like to customize your 404 page instead of using the default, you may add either a \texttt{\_404.Rmd} or \texttt{\_404.md} file to your project root and use code and/or Markdown syntax.

\hypertarget{metadata-for-sharing}{%
\section{Metadata for sharing}\label{metadata-for-sharing}}

Bookdown HTML books will provide HTML metadata for social sharing on platforms like Twitter, Facebook, and LinkedIn, using information you provide in the \texttt{index.Rmd} YAML. To setup, set the \texttt{url} for your book and the path to your \texttt{cover-image} file. Your book's \texttt{title} and \texttt{description} are also used.

This \texttt{gitbook} uses the same social sharing data across all chapters in your book- all links shared will look the same.

Specify your book's source repository on GitHub using the \texttt{edit} key under the configuration options in the \texttt{\_output.yml} file, which allows users to suggest an edit by linking to a chapter's source file.

Read more about the features of this output format here:

\url{https://pkgs.rstudio.com/bookdown/reference/gitbook.html}

Or use:

\begin{Shaded}
\begin{Highlighting}[]
\NormalTok{?bookdown}\SpecialCharTok{::}\NormalTok{gitbook}
\end{Highlighting}
\end{Shaded}


  \bibliography{book.bib,packages.bib}

\end{document}
